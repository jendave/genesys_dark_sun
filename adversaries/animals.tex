\subsection{Household Animals}


Often small creatures are kept or allowed to live in Athasian
households in exchange for some form of comfort or service the
creature provides. These pets are considered neither cute nor
cuddly as no such creature may be found anywhere on Athas.

\subsubsection{Hurrum}

These brightly-colored beetles are highly prized for the pleasant
humming sounds they produce. Better trading houses have at
least one.

The hurrum have a brightly-colored, opalescent, chitinous
shell that varies from cobalt-blue to copper-green. Convex and
oval in shape, the shell protects four pair of small, vestigial
wings. With a smallish head, stubby antennae, and four very
short legs, the beetle appears comical.\\
Hurrum beat their wings rapidly back and forth, gently striking
the underside of their carapace which creates the vibration
and noise for which these creatures are best known. The sound
is also used as a simple form of communication between hurrum
beetles.

\paragraph{Ecology}
This flightless beetle produces soothing humming noises
(although few would call it music) which change in pitch and
frequency throughout the day. Frequently, the bright, opalescent
shell of the hurrum beetle changes color with the speed of its
humming. The humming is a by-product of the creatures attempt to
cool itself by rapidly beating its small, vestigial wings. A real
pleasure is letting the beetle crawl across ones bare skin where the
slight vibration of the shell and the humming are coupled with a
very slight breeze. Halflings find hurrum extremely pleasing
(though when hungry, halflings find them delicious).

\subsubsection{Critic}

Multi-colored, spiny-backed lizards, critics are frequently reluctant
house guests in Athas. They are innately psionic and tune
themselves to their feeders.\\
\\
Some say critics are the prettiest lizards on Athas. Often mottled
in brightly-colored hues, they change color each year when
they molt. Critics average about 40cm length and weigh about 2kg.\\
\\
Critics use rudimentary body language to communicate with
others of their species. Communication via magical or other
means is possible; however, the creature often reacts/answers in
a paranoid or anxious manner.

\paragraph{Ecology}
Critics are considered to be good luck in a household
or storage area. They are lazy and prefer to be fed rather than
hunt on their own. Generally young critics are captured and
brought to a residence. They must be allowed to adjust to their
new surroundings at their own pace. Within a week, the feeder/
owner will know if the creature has decided to adopt the
location or has run away. The critic will remain in a constant
state of alarm if caged or chained.

\subsubsection{Renk}

This small gastropod has developed a symbiotic relationship
with humanoid creatures in the desert. A harmless, tasteless
slug, it stores water and is sometimes consumed raw on long
desert trips.\\
Varied in color, renk have a 2-3 elongated, tapered body. A
sucker mouth can be found below a short pair of antennae used
to detect vibration.

\paragraph{Ecology}
Renk are often taken on long trips. Consumed alive,
renk contain more water than seems possible. Renk store moisture
in an extra stomach that ruptures when they are eaten raw.
An average renk holds 1⁄2 cup of water; therefore, an active
man would need to eat 32 raw renks a day to replace fluids needed
for one day in the desert. Concern should be taken when purchasing
renk. Occasionally a leech or other harmful creature
will be added to a group of renk and sold to an unsuspecting
buyer.

\subsubsection{Ock'n}

Ock'n appear as small, spiral-shelled snails. When they move,
they leave a slime trail composed of an amber-like liquid that
has many household uses.\\
\\
Nautiloid in shape, this small snail seldom reaches 1 in
length. Ock'n shells are almost always light in hue, but vary in
color and striping. All ockn sensory organs are located on the
forward protruding head. The head holds a very small pair of
light sensors affixed to independent eye stalks. A slit mouth is
also located on the head.\\
\\
Ock'n communicate to other gastropods via a complex sys-
tem of eye stalk movements. The eye stalks are always in a
slow, constant motion as the creature conveys only the most
simple of concepts.

\paragraph{Ecology}
Unpalatable as food, ockn shells make interesting jewelry.
The gastropod spends each day in search of food and moisture,
leaving behind a valuable slime trail. The glossy,
ochre-colored slime is composed of an amber-like resin. When
the resin dries, it becomes hard as stone (saves as stone also),
but has only a quarter of normal stones weight. The snails
highly-prized slime/resin is used in numerous ways. Coating
weapons, waterproofing materials, and sealing perishables or
the dead are just a few of its uses. Ock'n snails only produce a
single fluid dram (1/8 fluid ounce) of this material each day.

\subsection{Herd Animals}
Many animals are used on Athas for food, clothing, or weap-
ons. Some have been partially domesticated or are stupid
enough to allow themselves to be captured and herded.

\subsubsection{Kip}

The kip is a shy, six-legged, armored creature that digs and eats
the roots of plants and trees. Their supple, armor-like covering
makes excellent durable leather goods.\\
\\
The leathered kip grows about 2.5m in length. Kips are covered
with a horny armor that is segmented into nine separate plates.
They have elongated, pointed snouts and very small, beady
eyes located on each side of their snouts. The eyes are protected
by a glass-like covering that protects them while digging. The
sharp, strong foreclaws enable the creature to dig. Their short,
stubby, hind legs only allow them to amble around slowly.
Kips communicate via a series of low grunts. Although not a
true language, dwarven kip-herders have learned to mimic the
sounds in order to better control the herd. Kip language can be
learned at a cost of a single non-weapon proficiency slot. As
with other languages, a simple check should be made against a
characters Intelligence for successful communication. Failure
means: 1) the kip doesnt understand and ignores the attempted
communication; 2) the wrong information is conveyed.

\paragraph{Ecology}
Kip herds provide a staple of meat and leather goods in
most dwarven communities. Roasted, an adult kip will feed two
very hungry dwarves; made into a stew, the meat and broth will
easily feed six. Kips and dwarves seem to be complementary in
pace and temperament. Dwarven communities keep the kip
herd in a slow, constant motion since kip tend to destroy roots
of growing plants. Their elongated, pointy snouts and keen
sense of smell aid them in their search for food. Kips will eat
anything that doesnt put up a fight, including garbage.
Dwarves with animal husbandry skill can safely milk the
pheromones from an adult, but the chemical will lose all potency
within 48 hours. Only a small amount of pheromonal fluid
can be milked from each animal. A turn spent milking a kip will
produce enough fluid to create a small (one cubic foot) pheromone
cloud.

\subsubsection{Z'tal}
Z'tal are small, upright lizards that hop in shepherded leaps
across Athas. They are very stupid and are known to stampede
when panicked.\\
\\
Z'tal jump wherever they go on powerful hind legs that end in
sharp talons. Their small head is rounded in the back with a
sharp, hard point on the end. Their long, thick tail is used for
balance when they hop. The vestigial forearms and claws are
seldom used. Varying in shades of tans and browns, ztal are
covered by sharp, feather-like scales.\\
\\
Z'tal constantly make a series of chirps, squeaks, and
squawks. These noises are intended predominantly to keep the
herd within a single area. When threatened, ztal scream and
run.

\paragraph{Ecology}
Z'tal meat is dense and coarse, though the hind quar-
ters (drumsticks) and tail of an adult ztal make excellent eating
(once the scales have been carefully removed). Roasted, the
three pieces will feed six hungry individuals or one half-giant.
Z'tal meat makes an excellent base for soup; a single ztal
cooked this way will easily feed a dozen hungry, man-sized
creatures. The sharp, flexible feather-scales are frequently
used as small knives and razors. The scales dull after a week
of regular use and can not be resharpened.

\subsubsection{Jankx}
These furred mammals live in burrow communities in the desert.
Although they represent a possible prime source of food or
clothing, most people think they are too dangerous to bother.
Standing about 1 tall on their hind legs, jankx have a small,
pointy head and internal cheek pouches that allow them to carry
food or water when they run. Jankx have long, sleek bodies
and four short, muscular legs. Golden in color, their pelts are
highly prized for trade.\\
\\
Jankx communicate in a series of ultrasonic squeaks and
barks that are inaudible to humanoid ears.

\paragraph{Ecology}
Those who consider themselves jankx herders are
more trappers than anything. Jankx are usually snared when
they are on their nightly forays. A herder hovel is always built
either above ground or on solid stone to prevent jankx from
burrowing underneath the hovel and killing the herder. Jankx
meat is considered gamey but palatable. Due to the difficult
position of the poison sacs, caution should be taken in preparing
jankx meat. A general rule is one jankx for elves and humans,
more for others. Caution must always be exercised when cooking
jankx meat. When cooked, the burning flesh emits an odor
that sends any nearby jankx into a rage. If cooked too close to a
jankx community, a horde of jankx will invariably find the
source of the odor and attack the diners.
