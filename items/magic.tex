\subsection{Magical and Psionics Gear}
\begin{multicols}{2}

\begin{table}[H]
\begin{GenesysTable}{Magic and Psionic Gear}{magic-and-psionic-gear}{ =X +l +l +l}
Name                                         & Encumbrance & Price & Rarity \\
\nameref{itmmgc:healingfruit}                & 0           & 30    & 5 \\
\nameref{itmmgc:spellcomponentpouch}         & 1           & 50    & (I) 3 \\
\end{GenesysTable}
\end{table}

\paragraph{Healing Fruit} \label{itmmgc:healingfruit}
Healing Fruits are fruit grown from magical infused fruit trees.
While using any fruit is possible, often pears are most common.
While they are great for infusing a living body with healing,
their effects do diminish quite fast. The first healing fruit
eaten on a day grants 5 healing, the second eaten the same day
4 healing and so forth.

\paragraph{Spell Component Pouch}
\label{itmmgc:spellcomponentpouch}
A spell component pouch contains the spell components neccesary to cast Primal
Spells. A spell component pouch can contain one or more components, which must
be bought seperately or be sought out. An spell component is empty when a \despair
is used to 'damage' the spell component using a magic skill check, in which case
the spell component must either be bought again or sought after anew. See~\tableref{spell-components} for a list for Spell Components.

\textit{TODO: Create better spell component names, and are they balanced?}

\begin{table*}[htb]
\begin{GenesysTable}{Spell Components}{spell-components}{ =X +l +l +l}
Component   & Cost      & Rarity    & Description \\
S1          & (I) 800   & 7         & When casting an conjure spell to summon an elemental,
                                            adding the Summon Ally effect does not increas its
                                            difficulty. In addition, the creature remains
                                            summoned untill the end of the encounter without
                                            your character having to use a concentrate manouvre.\\
S2          & (I) 400   & 6         & When casting a spell, adding the first Range effect added
                                            to the spell does not increase the spell's difficulty.\\
S3          & (I) 1000  & 9         & When casting a spell, the caster may count any additional
                                            \success beyond those needed to hit as \advantage\advantage\advantage
                                            required to activate any added effect.\\
%S4          &          & Rarity     & Attack spells cast by the user increase their base damage by 4.\\
\end{GenesysTable}
\end{table*}

\end{multicols}
