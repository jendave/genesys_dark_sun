\subsection{Magical and Psionics Gear}
\begin{multicols}{2}

\begin{table}[H]
\centering
\small\caption{Magic and Psionic Gear}
\begin{GenesysTable}{X l l l}
Name                                         & Encumbrance & Price & Rarity \\
\nameref{itmmgc:healingfruit}                & 0           & 30    & 5 \\
\nameref{itmmgc:spellcomponentpouch}         & 1           & 50    & 3 \\
\nameref{itmmgc:impspellcomponentpouch}      & 1           & 500   & 5 \\
\nameref{itmmgc:advancedspellcomponentpouch} & 1           & 1000  & 8 \\
\end{GenesysTable}
\end{table}

\paragraph{Healing Fruit} \label{itmmgc:healingfruit}
Healing Fruits are fruit grown from magical infused fruit trees.
While using any fruit is possible, often pears are most common.
While they are great for infusing a living body with healing,
their effects do diminish quite fast. The first healing fruit
eaten on a day grants 5 healing, the second eaten the same day
4 healing and so forth.

\paragraph{Spell Component Pouch}
\label{itmmgc:spellcomponentpouch}
A spell component pouch contains the spell components neccesary to cast Arcana and Primal Spells.

\paragraph{Improved Spell Component Pouch}
\label{itmmgc:impspellcomponentpouch}
An improved spell component pouch contains components which can help with or amplify spells
when cast using those components. When it is empty it reverts back to being a \nameref{itmmgc:spellcomponentpouch}.
If you have an \nameref{itmmgc:spellcomponentpouch} you can upgrade it to an Improved Spell
Component Pouch by paying the difference.

%TODO: Create better spell component names
\begin{table}[H]
\centering
\small\caption{Improved Spell Component Pouch}
\begin{GenesysTable}{l X}
Component   & Description \\
S1          & When casting an conjure spell to summon an elemental,
                adding the Summon Ally effect does not increas its
                difficulty. In addition, the creature remains
                summoned untill the end of the encounter without
                your character having to use a concentrate manouvre.\\
S2          & When casting a spell, adding the first Range effect added
                to the spell does not increase the spell's difficulty.\\
S3          & Attack spells cast by the user increase their base damage by 4.\\
S4          & When casting a spell, the caster may count any additional
                \success beyond those needed to hit as \advantage\advantage\advantage
                required to activate any added effect.\\
\end{GenesysTable}
\end{table}

\paragraph{Advanced Spell Component Pouch}
\label{itmmgc:advancedspellcomponentpouch}
Contains the same components as \nameref{itmmgc:impspellcomponentpouch}, in addition to the following:

\begin{table}[H]
\centering
\small\caption{Advanced Spell Component Pouch}
\begin{GenesysTable}{l X}
Component   & Description \\
\end{GenesysTable}
\end{table}

When it is damaged or empty, it reverts back to being a \nameref{itmmgc:impspellcomponentpouch}.
If you have an \nameref{itmmgc:impspellcomponentpouch} you can upgrade it to an Advanced Spell
Component Pouch by paying the difference.

\end{multicols}
