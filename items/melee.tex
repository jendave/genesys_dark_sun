\subsection{Melee Weapons}

\fxnote{\currentname: Check the balance of these weapons}

\begin{table*}[!htb]
\begin{GenesysTable}{Brawl Weapons}{brawl-weapons}{ =l +l +l +l +l +l +X}
Name                            & Dam & Crit & Encum & Price   & Rarity & Special     \\
\nameref{itmmlee:punchik}       & +1  & 3    & 1     & 75 cp   & 0      & \iqtyref{pierce} 1 \\
\nameref{itmmlee:talid}         & +0  & 4    & 1     & 60 cp   & 1      & \iqtyref{backup}, \iqtyref{disorient} 3 \\
\nameref{itmmlee:wristrazors}   & +1  & 3    & 0     & 30 cp   & 2      & \iqtyref{backup}, \iqtyref{pierce} 2 \\
\nameref{itmmlee:net}           & -1  & -    & 1     & 30 cp   & 4      & \iqtyref{ensnare} 4, \iqtyref{thrown} \\
\nameref{itmmlee:whip}          & +1  & 4    & 0     & 30 cp   & 5      & \iqtyref{ensnare} 2, \iqtyref{stundamage} \\
\nameref{itmmlee:chatrangswhip} & +2  & 3    & 1     & 300 cp  & 7      & \iqtyref{ensnare} 2, \iqtyref{vicious}, \iqtyref{inaccurate}, \iqtyref{stundamage} \\
\nameref{itmmlee:tortoiseblade} & +0  & 6    & 1     & 90 cp   & 3      & \iqtyref{defensive} 1, \iqtyref{inaccurate} 2, \iqtyref{pierce} 1 \\
\end{GenesysTable}
\end{table*}

\begin{table*}[!htb]
\begin{GenesysTable}{Light Melee Weapons}{light-melee-weapons}{ =l +l +l +l +l +l +X}
Name                         & Dam & Crit & Encum & Price    & Rarity & Special  \\
\nameref{itmmlee:knife}      & +1  & 3    & 1     & 25 cp    & 1   & \iqtyref{thrown} \\
\nameref{itmmlee:buckler}    & +0  & 6    & 1     & 40 cp    & 0   & \iqtyref{defensive} 1, \iqtyref{inaccurate} 1 \\
\nameref{itmmlee:club}       & +2  & 5    & 2     & 15 cp    & 1   & \iqtyref{disorient} 4 \\
\nameref{itmmlee:shortspear} & +2  & 4    & 2     & 90 cp    & 1   & \iqtyref{accurate} 1, \iqtyref{defensive} 1, \iqtyref{thrown} \\
\nameref{itmmlee:carrikal}   & +3  & 3    & 2     & 150 cp   & 3   & \iqtyref{vicious} 1 \\
\nameref{itmmlee:alhulak}    & +2  & 4    & 2     & 90 cp    & 2   & \iqtyref{disarm} \\
\nameref{itmmlee:macuahuitl} & +1  & 2    & 2     & 200 cp   & 5   & \iqtyref{vicious} 1 \\
\end{GenesysTable}
\end{table*}

\begin{table*}[!htb]
\begin{GenesysTable}{Heavy Melee Weapons}{heavy-melee-weapons}{ =l +l +l +l +l +l +X}
Name                          & Dam & Crit  & Encum & Price     & Rarity & Special     \\
\nameref{itmmlee:shield}      & +0  & 6     & 2     & 80 cp     & 1     & \iqtyref{defensive} 1, \iqtyref{deflection} 1, \iqtyref{inaccurate} 1, \iqtyref{knockdown} \\
\nameref{itmmlee:gouge}       & +4  & 2     & 3     & 300 cp    & 4     & \iqtyref{disorient} 2, \iqtyref{unwieldy} 3 \\
\nameref{itmmlee:longspear}   & +3  & 4     & 3     & 250 cp    & 2     & \iqtyref{reach}, \iqtyref{defensive} 1, \iqtyref{pierce} 1 \\
\nameref{itmmlee:lotulis}     & +4  & 3     & 4     & 300 cp    & 4     & \iqtyref{cumbersome} 3, \iqtyref{pierce} 1, \iqtyref{sunder} \\
\nameref{itmmlee:trikal}      & +3  & 3     & 5     & 250 cp    & 2     & \iqtyref{defensive} 1, \iqtyref{pierce} 3 \\
\nameref{itmmlee:cahulaks}    & +2  & 3     & 2     & 240 cp    & 3     & \iqtyref{thrown}, \iqtyref{reach}, \iqtyref{unwieldy} 3 \\
\nameref{itmmlee:dragonpaw}   & +2  & 4     & 2     & 315 cp    & 1     & \iqtyref{defensive} 1, \iqtyref{accurate} 1, \iqtyref{disarm} \\
\nameref{itmmlee:gythka}      & +2  & 4     & 2     & 250 cp    & 4     & \iqtyref{thrown}, \iqtyref{vicious} 1 \\
\end{GenesysTable}
\end{table*}
\begin{multicols}{2}

\subsubsection{Alhulak}
\label{itmmlee:alhulak}
This weapon is an unusual flail. A short length of rope separates
a four-bladed, hafted grappling hook from the handle.

\subsubsection{Buckler}
\label{itmmlee:buckler}
Whether crafted from wood, chitin, or hide, shields are
common among warriors of all cultures and skill levels for
a simple reason: they keep you alive. The utility of a shield
for blocking and parrying blows cannot be overstated. While
an important part in every warriors defence, the scorching sun
makes carrying a large shield impracticle and thus use medium
shields or even bucklers.

\subsubsection{Cahulaks}
\label{itmmlee:cahulaks}
Cahulaks are a pair of four-bladed weapons
held together with a length of rope.
They can be used in each hand as melee
weapons; one or both can also be thrown to
tangle and cause damage to an opponent.
The blades are commonly carved from
the hip or shoulder bones of a mekillot,
but more expensive versions can be forged
of steel. The hafts are made of solid
lengths of wood or, rarely, sturdy bone.
The connecting rope is up to 30 cm long; an
experienced cahulak wielder keeps most
of that length looped loosely in one hand
when preparing for combat.

\subsubsection{Carrikal}
\label{itmmlee:carrikal}
This axe has two forward-facing blades carved from the front
of a large jawbone, commonly that of a mekillot.

\subsubsection{Club}
\label{itmmlee:club}
This weapon is usually just a shaped piece of wood, sometimes
with a few stone or obsidian shards embedded in it.

\subsubsection{Dragon Paw}
\label{itmmlee:dragonpaw}
The dragon's paw is a multibladed weapon
popular among the arena masters of
Urik and Tyr. The weapon has two blades,
made from any material, one at each end
of a 12-15 cm wooden shaft. Around the center
is a bar or basket that both protects the
hand and holds another blade jutting
perpendicular to the central shaft. This
blade is called the forward blade, while
the others are called the outer blades.

\subsubsection{Gouge}
\label{itmmlee:gouge}
The shoulder-strapped gouge is a specialized
infantry weapon perfected for the
slave armies of the Shadow King of Nibenay.
It is a weapon that can inflict significant
damage against an opponent and is
unlikely to be dropped in the event of a
rout. The gouge itself has a wide bone,
obsidian, or chitin blade mounted onto a
3 long wooden shaft. A smaller handle
protrudes from a forward position on the
main shaft, while the rear of the shaft has
a wide grip used to drive the weapon
home. The shoulder strap is made of leath-
er or cloth, and it sometimes is expanded
to a complete harness around the neck
and shoulders. The weapon can be easily
turned over to accommodate a left-handed
wielder.

\subsubsection{Gythka}
\label{itmmlee:gythka}
Each end of this thri-kreen staff has a
small, crescent-shaped blade with a centered stabbing
tine. The secondary end of this double weapon is light
enough to be used as an off-hand weapon. A gythka
can be thrown like a javelin.

\subsubsection{Knife}
\label{itmmlee:knife}
A dagger has an obsidian blade that is about 30 cm in length.

\subsubsection{Long Spear}
\label{itmmlee:longspear}
Although a simple weapon, a spear is easy to wield and
allows the user to keep some distance from an oppo-
nent. Hence, spears don’t have very high damage, but
the Accurate 1 quality represents their ease of use. In
addition, the Defensive 1 quality represents their use-
fulness at keeping someone at arms’ reach.

\subsubsection{Lotulis}
\label{itmmlee:lotulis}
This short-staffed double weapon sports outward-pointing,
barbed crescent blades on each end.

\subsubsection{Macuahuitl}
\label{itmmlee:macuahuitl}
A macuahuitl is a wooden club with obsidian blades. Its sides
are embedded with prismatic blades traditionally made from
obsidian. The macuahuitl was a standard close combat weapon.

\subsubsection{Net}
\label{itmmlee:net}
A net is a web of rope or cord fitted with heavy weights.

\subsubsection{Punchik}
\label{itmmlee:punchik}
Punhicks are often little more than a wooden or bone
bar with a large obsidian spike attached to it, with the spike
coming out between your fingers. They are the smallest, simplest,
and easiest to conceal type of brawl weapon. Due to their
small size, punchiks are quite easy to conceal in a pocket,
pouch, or compartment in easy reach until they're needed. Add
\setback to a character's Perception check when attempting to
find a punchik on a person's body.

\subsubsection{Shield}
\label{itmmlee:shield}
Whether crafted from wood, chitin, or hide, shields are
common among warriors of all cultures and skill levels for
a simple reason: they keep you alive. The utility of a shield
for blocking and parrying blows cannot be overstated. While
an important part in every warriors defence, the scorching sun
makes carrying a large shield impracticle and thus use medium
shields or even bucklers.

\subsubsection{Short Spear}
\label{itmmlee:shortspear}
Although a simple weapon, a spear is easy to wield and
allows the user to keep some distance from an oppo-
nent. Hence, spears don’t have very high damage, but
the Accurate 1 quality represents their ease of use. In
addition, the Defensive 1 quality represents their use-
fulness at keeping someone at arms’ reach.

\subsubsection{Talid}
\label{itmmlee:talid}
Made from leather, chitin, and bone, this spiked
"gladiator's gauntlet" augments unarmed attacks.

\subsubsection{Tortoise Blade}
\label{itmmlee:tortoiseblade}
This bony or chitinous plate is affixed with a short
blade that points forward from the wielder's hand.

\subsubsection{Trikal}
\label{itmmlee:trikal}
This polearm projects three blades symmetrically lengthwise
from its haft. A trikal is equivalent to a halberd.

\subsubsection{Whip}
\label{itmmlee:whip}
Although a whip is impractical as a weapon
in most circumstances, some opponents are prone
to underestimating the wielder of a whip, which can
lead them to attack rashly or make other mistakes.

\subsubsection{Chatrangs Whip}
\label{itmmlee:chatrangswhip}
These whips are made from the tethers from a Chatrangs. It
is a vicious weapon, used by some Templars but forbidden to
be used by anyone else.

\subsubsection{Wrist Razors}
\label{itmmlee:wristrazors}
This weapon consists of three sharp blades that protrude from
a sturdy bracer, freeing the wielder's hand. A shield cannot
be worn on the same arm as wrist razors. Wrist razors do not
need to be drawn, nor do they need to be sheathed for the
wielder to use the hand the razors are on

\end{multicols}
