%TODO: Rebalance
\subsection{Melee Weapons}

\begin{table*}[!htb]
\centering
\small\caption{Brawl Weapons}
\begin{GenesysTable}{l l l l l l X}
Name                            & Dam & Crit & Encum & Price   & Rarity & Special     \\
\nameref{itmmlee:punchik}       & +1  & 3    & 1     & 75 cp   & 0      & \nameref{iqty:fragile}, \nameref{iqty:pierce} 1 \\
\nameref{itmmlee:talid}         & +0  & 4    & 1     & 60 cp   & 2      & \nameref{iqty:fragile}, \nameref{iqty:backup}, \nameref{iqty:disorient} 3 \\
\nameref{itmmlee:wristrazors}   & +1  & 3    & 0     & 30 cp   & 2      & \nameref{iqty:fragile}, \nameref{iqty:backup}, \nameref{iqty:pierce} 2 \\
\nameref{itmmlee:net}           & -1  & -    & 1     & 30 cp   & 2      & \nameref{iqty:ensnare} 4, \nameref{iqty:thrown} \\
\nameref{itmmlee:whip}          & +1  & 4    & 0     & 30 cp   & 2      & \nameref{iqty:ensnare} 2, \nameref{iqty:stundamage} \\
\nameref{itmmlee:tortoiseblade} & +0  & 6    & 1     & 90 cp   & 3      & \nameref{iqty:fragile}, \nameref{iqty:defensive} 1, \nameref{iqty:inaccurate} 2, \nameref{iqty:pierce} 1 \\
\end{GenesysTable}
\end{table*}

\begin{table*}[!htb]
\centering
\small\caption{Light Melee Weapons}
\begin{GenesysTable}{l l l l l l X}
Name                         & Dam & Crit & Encum & Price    & Rarity & Special  \\
\nameref{itmmlee:knife}      & +1  & 3    & 1     & 25 cp    & 1   & \nameref{iqty:fragile}, \nameref{iqty:thrown} \\
\nameref{itmmlee:buckler}    & +0  & 6    & 1     & 40 cp    & 0   & \nameref{iqty:defensive} 1, \nameref{iqty:inaccurate} 1 \\
\nameref{itmmlee:club}       & +2  & 5    & 2     & 15 cp    & 1   & \nameref{iqty:disorient} 4 \\
\nameref{itmmlee:shortspear} & +2  & 4    & 2     & 90 cp    & 1   & \nameref{iqty:fragile}, \nameref{iqty:accurate} 1, \nameref{iqty:defensive} 1, \nameref{iqty:thrown} \\
\nameref{itmmlee:carrikal}   & +3  & 3    & 2     & 150 cp   & 1   & \nameref{iqty:fragile}, \nameref{iqty:vicious} 1 \\
\nameref{itmmlee:alhulak}    & +2  & 4    & 2     & 90 cp    & 1   & \nameref{iqty:fragile}, \nameref{iqty:disarm} \\
\end{GenesysTable}
\end{table*}

\begin{table*}[!htb]
\small\caption{Heavy Melee Weapons}
\centering
\begin{GenesysTable}{l l l l l l X}
Name                          & Dam & Crit  & Encum & Price     & Rarity & Special     \\
\nameref{itmmlee:shield}      & +0  & 6     & 2     & 80 cp     & 1     & \nameref{iqty:defensive} 1, \nameref{iqty:deflection} 1, \nameref{iqty:inaccurate} 1, \nameref{iqty:knockdown} \\
\nameref{itmmlee:gouge}       & +4  & 2     & 3     & 300 cp    & 4     & \nameref{iqty:fragile}, \nameref{iqty:disorient} 2, \nameref{iqty:unwieldy} 3 \\
\nameref{itmmlee:longspear}   & +3  & 4     & 3     & 250 cp    & 2     & \nameref{iqty:fragile}, \nameref{iqty:reach}, \nameref{iqty:defensive} 1, \nameref{iqty:pierce} 1 \\
\nameref{itmmlee:lotulis}     & +4  & 3     & 4     & 300 cp    & 4     & \nameref{iqty:fragile}, \nameref{iqty:cumbersome} 3, \nameref{iqty:pierce} 1, \nameref{iqty:sunder} \\
\nameref{itmmlee:trikal}      & +3  & 3     & 5     & 250 cp    & 2     & \nameref{iqty:fragile}, \nameref{iqty:defensive} 1, \nameref{iqty:pierce} 3 \\
\nameref{itmmlee:cahulaks}    & +2  & 3     & 2     & 275 cp    & 2     & \nameref{iqty:fragile}, \nameref{iqty:thrown}, \nameref{iqty:reach}, \nameref{iqty:unwieldy} 3 \\
\nameref{itmmlee:dragonpaw}   & +2  & 4     & 2     & 275 cp    & 2     & \nameref{iqty:fragile}, \nameref{iqty:defensive} 1, \nameref{iqty:accurate} 1, \nameref{iqty:disarm} \\
\nameref{itmmlee:gythka}      & +2  & 4     & 2     & 275 cp    & 2     & \nameref{iqty:fragile}, \nameref{iqty:thrown}, \nameref{iqty:vicious} 1 \\
\end{GenesysTable}
\end{table*}
\begin{multicols}{2}

\subsubsection{Alhulak}
\label{itmmlee:alhulak}
This weapon is an unusual flail. A short length of rope separates
a four-bladed, hafted grappling hook from the handle.

\subsubsection{Buckler}
\label{itmmlee:buckler}
Whether crafted from wood, chitin, or hide, shields are
common among warriors of all cultures and skill levels for
a simple reason: they keep you alive. The utility of a shield
for blocking and parrying blows cannot be overstated. While
an important part in every warriors defence, the scorching sun
makes carrying a large shield impracticle and thus use medium
shields or even bucklers.

\subsubsection{Cahulaks}
\label{itmmlee:cahulaks}
This double weapon features two four-
bladed, hafted heads separated by a length of rope.
The secondary end is light enough to be used as an
off-hand weapon. When one end of this weapon is
held by the haft, the rope is long enough to grant the
other end reach. The entire weapon can be thrown.

\subsubsection{Carrikal}
\label{itmmlee:carrikal}
This axe has two forward-facing blades carved from the front
of a large jawbone, commonly that of a mekillot.

\subsubsection{Club}
\label{itmmlee:club}
This weapon is usually just a shaped piece of wood, sometimes
with a few stone or obsidian shards embedded in it.

\subsubsection{Dragon Paw}
\label{itmmlee:dragonpaw}
Short blades attach to either end
of this staff. In the center of this double weapon is a
guard with a protruding blade perpendicular to the
staff. The light, middle blade (which serves as the
off-hand end) can be used for quick jabs, ideal for a
warrior with a roguish bent.

\subsubsection{Gouge}
\label{itmmlee:gouge}
This spadelike weapon has a long haft with a handle on the end. The head
is a wide, double-edged blade with a stabbing point at the top. Some
gouges are fitted with a strap or a harness, making the weapon easier
to carry.

\subsubsection{Gythka}
\label{itmmlee:gythka}
Each end of this thri-kreen staff has a
small, crescent-shaped blade with a centered stabbing
tine. The secondary end of this double weapon is light
enough to be used as an off-hand weapon. A gythka
can be thrown like a javelin.

\subsubsection{Knife}
\label{itmmlee:knife}
A dagger has an obsidian blade that is about 30 cm in length.

\subsubsection{Long Spear}
\label{itmmlee:longspear}
Although a simple weapon, a spear is easy to wield and
allows the user to keep some distance from an oppo-
nent. Hence, spears don’t have very high damage, but
the Accurate 1 quality represents their ease of use. In
addition, the Defensive 1 quality represents their use-
fulness at keeping someone at arms’ reach.

\subsubsection{Lotulis}
\label{itmmlee:lotulis}
This short-staffed double weapon sports outward-pointing,
barbed crescent blades on each end.

\subsubsection{Net}
\label{itmmlee:net}
A net is a web of rope or cord fitted with heavy weights.

\subsubsection{Punchik}
\label{itmmlee:punchik}
Punhicks are often little more than a wooden or bone
bar with a large obsidian spike attached to it, with the spike
coming out between your fingers. They are the smallest, simplest,
and easiest to conceal type of brawl weapon. Due to their
small size, punchiks are quite easy to conceal in a pocket,
pouch, or compartment in easy reach until they're needed. Add
\setback to a character's Perception check when attempting to
find a punchik on a person's body.

\subsubsection{Shield}
\label{itmmlee:shield}
Whether crafted from wood, chitin, or hide, shields are
common among warriors of all cultures and skill levels for
a simple reason: they keep you alive. The utility of a shield
for blocking and parrying blows cannot be overstated. While
an important part in every warriors defence, the scorching sun
makes carrying a large shield impracticle and thus use medium
shields or even bucklers.

\subsubsection{Short Spear}
\label{itmmlee:shortspear}
Although a simple weapon, a spear is easy to wield and
allows the user to keep some distance from an oppo-
nent. Hence, spears don’t have very high damage, but
the Accurate 1 quality represents their ease of use. In
addition, the Defensive 1 quality represents their use-
fulness at keeping someone at arms’ reach.

\subsubsection{Talid}
\label{itmmlee:talid}
Made from leather, chitin, and bone, this spiked
"gladiator's gauntlet" augments unarmed attacks.

\subsubsection{Tortoise Blade}
\label{itmmlee:tortoiseblade}
This bony or chitinous plate is affixed with a short
blade that points forward from the wielder's hand.

\subsubsection{Trikal}
\label{itmmlee:trikal}
This polearm projects three blades symmetrically lengthwise
from its haft. A trikal is equivalent to a halberd.

\subsubsection{Whip}
\label{itmmlee:whip}
Although a whip is impractical as a weapon
in most circumstances, some opponents are prone
to underestimating the wielder of a whip, which can
lead them to attack rashly or make other mistakes.

\subsubsection{Wrist Razors}
\label{itmmlee:wristrazors}
This weapon consists of three sharp blades that protrude from
a sturdy bracer, freeing the wielder's hand. A shield cannot
be worn on the same arm as wrist razors. Wrist razors do not
need to be drawn, nor do they need to be sheathed for the
wielder to use the hand the razors are on

\end{multicols}
