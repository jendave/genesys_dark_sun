\begin{table*}[!htb]
\centering
\small\caption{Augment Additional Effects}
\begin{GenesysTable}{l X}
Cost                    & Effect\\
\difficulty             & \textbf{Haste:} Targets affected by the spell can always perform
                            a second maneuver during their turn without spending
                            strain (they may still only perform two maneuvers a turn).\\
\difficulty             & \textbf{Fury:} The target adds damage equal to the character's
                            ranks in Knowledge to unarmed combat checks, and their
                            Critical rating for unarmed combat checks becomes 3.\\
\difficulty             & \textbf{Range:} Increase the range of the spell by one range band.
                            This may be added multiple times, increasing the range
                            by one range band each time.\\
\difficulty             & \textbf{Swift:} Targets affected by the spell ignore the effects
                            of difficult terrain and cannot be immobilized.\\
\difficulty\difficulty  & \textbf{Additional Target:} The spell affects one additional target
                            within range of the spell. In addition, after casting
                            the spell, you may spend \advantage to affect one
                            additional target within range of the spell (and may
                            trigger this multiple times, spending \advantage each time).\\
\end{GenesysTable}
\label{table:magic_augment}
\end{table*}

%TODO: rephrease text
%TODO: replace Knowledge
\subsubsection{Augment}
\textbf{Skill:} Primal\\
\textbf{Concentration:} Yes\\
This is using magic to enhance people. A character selects
one target they are engaged with (which can be themself),
then makes a Primal or Divine skill check. The default
difficulty of the check is Average (\difficulty\difficulty).
If the check is successful, until the end of your character's
next turn, the target increases the ability of any skill checks
they make by one (in effect, this means they add \proficiency to
their checks).
A character may not be affected by more than one Augment spell
at the same time (so no stacking effects).
Before making an augment check, you may choose any number of
additional effects from ~\tref{table:magic_augment}.
