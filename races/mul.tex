%TODO: rebalance
\Race[name=Mul, image=images/mul.png, brawn=3, presence=1, strain=14, xp=90] { \epigraph{\textit{
"See, the trick is to break their will. Not too much, mind
you. Nobody wants to watch a docile gladiator, and muls
are too expensive to waste as labor slaves. But, you don't
want them trying to escape every other day. Would you like
to tell the arena crowd that their favorite champion will not
be appearing in today's match because he died trying to
escape your pens?"
} }{
Gaal, Urikite arena trainer
}

Born from the unlikely parentage of dwarves and
humans, muls combine the height and adaptable nature of
humans with the musculature and resilience of dwarves.
Muls enjoy traits that are uniquely their own, such as
their robust metabolism and almost inexhaustible
capacity for work. The hybrid has disadvantages in a few
areas as well: sterility, and the social repercussions of
being created for a life of slavery. Humans and dwarves
are not typically attracted to each other. The only reason
that muls are so common in the Tablelands is because of
their value as laborers and gladiators: slave-sellers force-
breed humans and dwarves for profit. While mul-
breeding practices are exorbitantly lucrative, they are
often lethal to both the mother and the baby. Conception
is difficult and impractical, often taking months to
achieve. Even once conceived, the mul takes a full twelve
months to carry to term; fatalities during this period are
high. As likely as not, anxious overseers cut muls from the
dying bodies of their mothers.\\

\textbf{Personality:}
All gladiators who perform well in the
arenas receive some degree of pampered treatment, but
muls receive more pampering than others. Some mul
gladiators even come to see slavery as an acceptable part
of their lives. However, those that acquire a taste of
freedom will fight for it. Stoic and dull to pain, muls are
not easily intimidated by the lash. Masters are loath to
slay or maim a mul who tries repeatedly to escape,
although those who help the mul’s escape will be
tormented in order to punish the mul without damaging
valuable property. Once a mul escapes or earns his
freedom, slavery remains a dominant part of his life. Most
muls are heavily marked with tattoos that mark his
ownership, history, capabilities and disciplinary
measures. Even untattooed muls are marked as a
potential windfall for slavers: it is clearly cheaper to
'retrieve' a mul who slavers can claim had run away,
than to start from scratch in the breeding pits.\\

\textbf{Names:} Muls sold as laborers will have common slave
names. Muls sold as gladiators will often be given more
striking and exotic names. Draji names (such as Atlalak)
are often popular for gladiators, because of the Draji
reputation for violence. Masters who change their mul
slaves' professions usually change their names as well,
since it is considered bad form to have a gladiator with a
farmer's name, and a dangerous incitement of slave
rebellions to give a common laborer the name of a
gladiator.\\

\textbf{Roleplaying Suggestions:}
Born to the slave pens, you never knew love or
affection; the taskmaster's whip took the place of loving
parents. As far as you have seen, all of life's problems that
can be solved are solved by sheer brute force. You know
to bow to force when you see it, especially the veiled force
of wealth, power and privilege. The noble and templar
may not look strong, but they can kill a man with a word.
You tend towards gruffness. In the slave pits, you knew
some muls that never sought friends or companionship,
but lived in bitter, isolated servitude. You knew other
muls who found friendship in an arena partner or co-
worker. You are capable of affection, trust and friendship,
but camaraderie is easier for you to understand and
express - warriors slap each other on the shoulder after a
victory, or give their lives for each other in battle. You
don't think of that sort of event as "friendship" - it just
happens.
}
{Mulls begin the game with one rank in Resilience. They may not train Resilience above rank 2 during character creation. }
{\item \textbf{Tireless:} Mulls add \boost to any Resilience checks.}
{}
{}
{}
{}
{}
